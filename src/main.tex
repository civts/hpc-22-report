\documentclass[conference]{IEEEtran}

\IEEEoverridecommandlockouts
% The preceding line is only needed to identify funding in the first footnote. If that is unneeded, please comment it out.
\usepackage[utf8]{inputenc} % Required for including letters with accents
\usepackage{cite}
\usepackage{amsmath,amssymb,amsfonts,mathtools}
\usepackage{algorithmic}
\usepackage{graphicx}
\usepackage{tabularx}

%Graphics
\usepackage{epsfig}
\usepackage{svg}
\svgsetup{
    clean = true,
    inkscapepath = ../out/converted_svgs/,
    inkscapelatex = false,
    inkscapearea = nocrop,
    width = \columnwidth
}
\usepackage{tikz}
\usetikzlibrary{shapes.geometric,arrows,fit,matrix,positioning,trees}
\usepackage[hidelinks]{hyperref} % Make the links clickable

\usepackage{enumitem} % Required for manipulating the whitespace between and within lists

% Define the inline list
\newlist{inlinelist}{enumerate*}{1}
\setlist*[inlinelist,1]{%
    label=(\roman*),
}

\usepackage{cleveref}

\usepackage{textcomp}
\usepackage{xcolor}
\def\BibTeX{{\rm B\kern-.05em{\sc i\kern-.025em b}\kern-.08em
T\kern-.1667em\lower.7ex\hbox{E}\kern-.125emX}}


\begin{document}


\title{
    Parallel Implementation and Benchmarking of the Closest Pair of Points Algorithm
    \break
    \break
    \footnotesize{
        Project report for the High Performance Computing for Data Science course. \\
        A.Y. 2022-23. Prof. Sandro Luigi Fiore
    }
}

\author{\IEEEauthorblockN{Davide Guidolin}
    \IEEEauthorblockA{davide.guidolin@studenti.unitn.it}
    \and
    \IEEEauthorblockN{Giacomo Zanolli}
    \IEEEauthorblockA{giacomo.zanolli@studenti.unitn.it}
}

\maketitle


\begin{abstract}
    This is the report for the High Performance Computing for Data Science course project.

    The goal of the project was to implement a parallel solution for the \textit{closest pair of points} problem
    and evaluate it on the High Performance Computing cluster of the University of Trento.

    In the following sections we will present the problem and we will describe some serial solutions
    for it. Then we will present how we implemented a parallel solution and finally we will present and
    discuss the evaluation of the parallel solution on the HPC cluster.
\end{abstract}

% ~ 4 pages total
\section{Introduction}
\label{sec:introduction}

% what we are talking about
The \textit{closest pair of points} problem consists
in finding the two nearest points in a set, $S$, according to a given distance function, $d$\cite{closest_pair_definition}.
These points can live in a space with an arbirary number of dimensions
and the choice of the distance function is also arbitrary.
For the purposes of this research, we focus on the two-dimensional problem, using the Euclidean distance and integer coordinates.
With minimal adjustments, the code can be adapted to solve a higher dimensionality problem, to adopt a different distance function or to use, e.g., floating point coordinates.

% objective
This research aims to study the implications of running an algorithm to solve the closest pair of points problem in parallel on a cluster for High Performance Computing.

% structure of the paper
This paper is structured as follows:
\begin{inlinelist}
    \item \Cref{sec:problem_analysis} describes the state of the art for the closest pair of points problem.
    \item \Cref{sec:parallelization} treats how we produced our parallel implementation.
    \item \Cref{sec:final_discussion} provides an evaluation of the obtained results using benchmarking techniques and our conclusions on the project as a whole.
\end{inlinelist}
% ~page 1
\section{Problem Analysis}
\label{sec:problem_analysis}

% description of the problem
The na\"ive solution to the closest pair of points problem would be to take each point
in the plane and calculate the distance between that point and all
the other points. The time complexity for this solution is $\Theta(N^2)$,
where $N\coloneqq|S|$ is the cardinality of the set $S$.

% Better solutions have been proposed, in particular in 1975
% Rabin proposed a stocastic algorithm with an expected run time
% of $O(N)$.

In 1975, Shamos and Hoey proved that the lower bound for the problem is $\Omega(N * log(N))$\cite[§2, theorem 1]{closest_pair_definition}. An algorithm with said complexity appears
the following year from Bentley and Shamos \cite{divide_and_conq_3NlgN}.
% in a book by Preparata and Shamos \cite[§5.4]{preparata1993computational}.
A new version of that algortihm was proposed in 2006, reducing the time complexity by a factor of two \cite{ge2006improved}.

For the purposes of this research, we decided to implement the version of Bentley and Shamos, which falls into the best category of complexity whilst still keeping the base algorithm relatively simple, aiding the comprehension of the code.
% For this research, we chose to use a divide and conquer approach
% with $O(N*(logN)^2)$ time complexity. We chose it because
% divide and conquer is highly parallelizable so it will be easier
% to exploit the computational power of the HPC cluster.

\subsection{Algorithm Explanation}

In order to solve the closest pair of points problem, we apply the divide et impera approach described in \cite{divide_and_conq_3NlgN}.
Our function accepts as input an array of $N$ points.

As a first step, this array is sorted in ascending order according to the X coordinate -time complexity: $\Theta(N*log(N))$-.

If the length of the array is two, then the distance between these two points is the minimum one.

If the length of the array is three, we compute the three distances between the pairs of points and return the minimum one.

If the length is greater than three, we:
\begin{inlinelist}
    \item divide the points by splitting them in two halves of length $\lfloor \frac{N}{2} \rfloor$ and $\lceil \frac{N}{2} \rceil$,
    \item find the closest pair in each half by recursively calling the closest pair of points function on them,
    \item merge the two results to find the closest pair in the starting array. \label{step:merge}
\end{inlinelist}

Let us focus our attention on step \ref{step:merge}: the merge.

In order to combine the two partial results, we first determine which among the the two pairs has the smallest distance, let that be $\delta$.

Next, we condiser a vertical band of points of thickness $2\delta$ centered on the line we used to split the points -i.e., on the median of the X coordinates of the rightmost point of the left half and the leftmost point of the right half-.

After sorting the points in that band according to their Y coordinate, we search if they contain a pair closer than $\delta$.

To do so, we compute the distance between each point of the band and the seven points that come after it,
%fifteen/eight/sixteen https://youtu.be/xi-WF07rAQw?t=1673 https://archive.org/details/AlgorithmDesign1stEditionByJonKleinbergAndEvaTardos2005PDF/page/n253/mode/2up
updating our closest pair if we find a couple of points that are closer than the minimum distance\cite{lingqi}.

%(Decidere se spiegare meglio e/o aggiungere l'immagine)
 % ~page 1 
\section{Main steps towards parallelization}
\label{sec:parallelization}

\begin{itemize}
  \item Design of the parallel solution
  \item Implementation
  \item Benchmark on the HPC@UniTrento cluster
\end{itemize}

To move from serial to parallel we decided to split the ordered
points between all the processes, then each process will perform
the serial algorithm on the subset and will find its local best distance and
finally the distances will be merged and the boundary points will be checked
 % ~page 2 
\section{Benchmarking}

The benchmarks were run on the University of Trento's High Performance Computing cluster
%\footnote{\url{https://unitrento.service-now.com/unitrento?id=unitrento_service_offering_dettaglio&category_id=68561769c33785109a8d151ce0013147&sys_id=14bb84f0c398d5104cbb7055df013133}}
.

The following combinations of parameters were considered:

\begin{table}[ht]
    \centering
    \caption{Simulation parameters}
    \begin{tabular}{l|l}
        \hline
        Parameter                     & Range                                    \\
        \hline
        CPU(s)                        & {1, 2, 4, 8, 16}                         \\
        Node(s)                       & {1, 2, 4, 8, 16}                         \\
        Packing strategy              & {pack, scatter, pack:excl, scatter:excl} \\
        Input size (number of points) & {8, 5K, 100K, 1M, 5M, 10M, 15M }         \\
        \hline
    \end{tabular}
\end{table}

The jobs were always submitted asking for 1GB of RAM and with a maximum execution time set to two minutes by default, but increased to five if we were running more than 64 processes.


\subsection{Execution time}

We saw the execution time varying like abcd
\begin{table}[ht]
    \centering
    \caption{Aggregate simulation results}
    \begin{tabular}{l|l|l|l}
        % or this to span the entire width \begin{tabularx}{\columnwidth}{X|X|X|X}
        \hline
        Nodes & Cores & Input size & Runtime (seconds) \\
        \hline
        2     & 2     & 1M         & 10                \\
        \hline
    \end{tabular}
\end{table}

\subsection{Speedup}
The speedup is an important measure in the evaluation of a parallel algorithm.
It is defined as the ratio of the serial runtime for solving a problem to the time taken by the
parallel algorithm to solve the same problem on p processors:
\begin{equation}
    Speedup = \frac{T_{serial}}{T_{parallel}}
\end{equation}
A value of speedup lower than 1 means that the parallelization did not provide a gain in execution time
since the parallel algorithm takes more time to solve the problem than the serial algorithm.
Ideally we would like to have an infinite speedup, however any speedup measure greater than one would
mean that the parallelization has been successful.

By looking at Fig. \ref{fig:speedup} we can see that our implementation reaches a speedup of $4.5$ on the
$5M$ and $15M$ datasets using the \textit{scatter:excl} strategy and a speedup of $4$ using the other strategies.
The $10M$ dataset reaches a slightly lower speedup, while the $1M$ dataset reaches the lowest speedup, especially
with an high number of processes. This is due to the fact that,

\subsection{Efficiency}

\subsection{Scalability}
Our implementation exhibited {strong-weak} scalability. TODO: figure this out

 % ~page 3
\section{Final Discussion}
\label{sec:final_discussion}

\subsection{Conclusions}
In this research, we produced an implementation of the closest pair of points algorithm, parallelized it through the use of OpenMPI and benchmarked it on a High Performance Comupting cluster.
We setup Continuous Integration and Continuous Deployment pipelines that helped us not only speeding up the testing, but also catching bugs in the code.

Overall, the obtained algorithm shows moderate speedups.

\subsection{Assumptions}
We assumed that:
\begin{itemize}
    \item the coordinates of a point fit inside an \texttt{int32};
    \item the points live in a two-dymensional space;
    \item we use the Euclidean distance function.
\end{itemize}

\subsection{Future Works}
\label{subsec:future_works}
For sure, we'd like to know how an addition of the OpenMP parallelization strategy influences the outcomes of the tests.

More rigorous benchmarking, with distinctions from the various times of our program.

A comparison with \cite{wang2020parallel} would also be great, but we were not able to figure it out in the time allotted.

Possible Optimizations:
\begin{enumerate}
    \item Instead of 3 \verb|MPI_send|, we can have two by merging the one for transmitting the best pair and the one for transmitting the two band lenghts.
    \item Parallel I/O could be considered: Since the filesystem is shared, each process may individually read the first line of the input, which in our input format contains the total number of points. It could open the offset in the file where its first point is using its rank, and read exactly the portion of points it has to elaborate. This of course assumes that all points occupy a known, constant, size in the file. That can be achieved by representing each number with a fixed length string -e.g., `+0137`' or `-9850`' if the range is $]-10000,10000[$ and we encode the file in ASCII-. Another possibility should be to use binary formats such as CBOR~\cite{bormann2013cbor}.
\end{enumerate}
 % ~page 4

\bibliographystyle{ieeetr}
\bibliography{bibliography}

\onecolumn % Switch to one column layout
\setlength\itemsep{1em} % Restore item spacing in lists

\appendix
\subsection{Full Benchmark Results}

This appendix contains the graphs with the relation between the number of processes and, respectively, the speedup, efficiency and total run time.

\begin{figure}[!ht]
    \centering
    \includesvg{../results/imgs/Efficiency_pack.svg}
    \includesvg{../results/imgs/Efficiency_pack:excl.svg}
    \includesvg{../results/imgs/Efficiency_scatter.svg}
    \includesvg{../results/imgs/Efficiency_scatter:excl.svg}
    \caption{Efficiency and number of processes correlation}
\end{figure}

\begin{figure}[!ht]
    \centering
    \includesvg{../results/imgs/Speedup_pack.svg}
    \includesvg{../results/imgs/Speedup_pack:excl.svg}
    \includesvg{../results/imgs/Speedup_scatter.svg}
    \includesvg{../results/imgs/Speedup_scatter:excl.svg}
    \caption{Efficiency and number of processes correlation}
\end{figure}

\begin{figure}[!ht]
    \centering
    \includesvg{../results/imgs/Total_time_pack.svg}
    \includesvg{../results/imgs/Total_time_pack:excl.svg}
    \includesvg{../results/imgs/Total_time_scatter.svg}
    \includesvg{../results/imgs/Total_time_scatter:excl.svg}
    \caption{Efficiency and number of processes correlation}
\end{figure}

\pagebreak
\subsection{Additional Resources}
\begin{itemize}
      \item Source code repository: \texttt{\url{https://github.com/civts/parallel-closest-pair}}
            \label{link:source}
      \item Input datasets: \texttt{\url{https://drive.google.com/drive/folders/12EFktu1ARw6exKn5k1AvLlHtD1Rcnx9f}}
      \item Spreadsheet with the raw data from the benchmarks -multiple bencharks, see benchmark name column-: \texttt{\url{https://docs.google.com/spreadsheets/d/1pG-KbEGtl7QHyF0bolPf2sWN__m63soeGuircwflwV0}}
            \label{link:spreadsheet}
\end{itemize}

\end{document}
