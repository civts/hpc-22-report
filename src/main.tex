\documentclass[conference]{IEEEtran}

\IEEEoverridecommandlockouts
% The preceding line is only needed to identify funding in the first footnote. If that is unneeded, please comment it out.
\usepackage[utf8]{inputenc} % Required for including letters with accents
\usepackage{amsmath,amssymb,amsfonts,mathtools}
\usepackage{algorithmic}
\usepackage{graphicx}
\usepackage{tabularx}

%Graphics
\usepackage{epsfig}
\usepackage{svg}
\svgsetup{
    clean = true,
    inkscapepath = ../out/converted_svgs/,
    inkscapelatex = false,
    inkscapearea = nocrop,
    width = \columnwidth
}
\usepackage{tikz}
\usetikzlibrary{shapes.geometric,arrows,fit,matrix,positioning,trees}
\usepackage[hidelinks]{hyperref} % Make the links clickable

\usepackage{enumitem} % Required for manipulating the whitespace between and within lists

% Define the inline list
\newlist{inlinelist}{enumerate*}{1}
\setlist*[inlinelist,1]{%
    label=(\roman*),
}

\usepackage{cleveref}

\usepackage{textcomp}
\usepackage{xcolor}

%Bibliogaphy
\usepackage[style=ieee,sorting=nty]{biblatex}
\bibliography{bibliography}

%Redefines the \url command so that we can specify multiple ones (used in the bibliography)
% https://tex.stackexchange.com/questions/57865
\renewbibmacro*{url}{\printfield[url]{urlraw}}
\let\URL\url
\makeatletter
\def\url{\begingroup \catcode`\%=12\catcode`\#=12\relax\printurl}
\def\printurl#1{\@URL#1 \@nil\endgroup}
\def\@URL#1 #2\@nil{\URL{#1}\ifx\relax#2\relax \else; \url{#2\relax}\fi}
\makeatother


\begin{document}


\title{
    Parallel Implementation and Benchmarking of the Closest Pair of Points Algorithm
    \break
    \break
    \footnotesize{
        Project report for the High Performance Computing for Data Science course. \\
        A.Y. 2022-23. Prof. Sandro Luigi Fiore
    }
}

\author{\IEEEauthorblockN{Davide Guidolin}
    \IEEEauthorblockA{davide.guidolin@studenti.unitn.it}
    \and
    \IEEEauthorblockN{Giacomo Zanolli}
    \IEEEauthorblockA{giacomo.zanolli@studenti.unitn.it}
}

\maketitle


\begin{abstract}
    This is the report for the High Performance Computing for Data Science course project.

    The goal of the project was to implement a parallel solution for the \textit{closest pair of points} problem
    and evaluate it on the High Performance Computing cluster of the University of Trento.

    In the following sections we will present the problem and we will describe some serial solutions
    for it. Then we will present how we implemented a parallel solution and finally we will present and
    discuss the evaluation of the parallel solution on the HPC cluster.
\end{abstract}

% ~ 4 pages total
\section{Introduction}
\label{sec:introduction}

% what we are talking about
The \textit{closest pair of points} problem consists
in finding the two nearest points in a set, $S$, according to a given distance function, $d$\cite{closest_pair_definition}.
These points can live in a space with an arbirary number of dimensions,
and the choice of the distance function is also arbitrary.
For the purposes of this research, we focus on the two-dimensional problem, using the Euclidean distance and integer coordinates.
With minimal adjustments, the code can be adapted to solve a higher dimensionality problem, to adopt a different distance function or to use, e.g., floating point coordinates.

% objective
This study aims to evaluate the implications of running an algorithm to solve the closest pair of points problem in parallel on a cluster for High Performance Computing\footnote{Source code available in \Cref{link:source}}.

% structure of the paper
This report is structured as follows:
\begin{inlinelist}
    \item \Cref{sec:problem_analysis} describes the state of the art for the closest pair of points problem and the algorithm we chose to solve it.
    \item \Cref{sec:parallelization} treats how we adapted the algorithm design to produce our parallel implementation.
    \item \Cref{sec:benchmarking} describes how the benchmarking was structured and
    provides its most relevant results.
    \item \Cref{sec:final_discussion} contains our conclusions about the performance and the scalability of our solution, a list of its limitations, and possible future works.
\end{inlinelist}
% ~page 1
\section{Problem analysis}
\label{sec:problem_analysis}

% description of the problem
The na\"ive solution to this problem would be to take each point
in the plane and calculate the distance between that point and all
the other points. The time complexity for this solution is $\Theta(N^2)$,
where $N\coloneqq|S|$ is the cardinality of the set $S$.

% Better solutions have been proposed, in particular in 1975
% Rabin proposed a stocastic algorithm with an expected run time
% of $O(N)$.

In 1975, Shamos and Hoey proved that the lower bound for the problem is $\Omega(N * log(N))$\cite[§2, theorem 1]{closest_pair_definition}. An algorithm with said complexity appears
the following year from Bentley and Shamos \cite{divide_and_conq_3NlgN}.
% in a book by Preparata and Shamos \cite[§5.4]{preparata1993computational}.
A new version of that algortihm was proposed in 2006, reducing the time complexity by a factor of two \cite{ge2006improved}.

For the purposes of this research, we decided to implement the version of Bentley and Shamos, which falls into the best category of complexity whilst still keeping the base algorithm relatively simple, aiding the comprehension of the code.
% For this research, we chose to use a divide and conquer approach
% with $O(N*(logN)^2)$ time complexity. We chose it because
% divide and conquer is highly parallelizable so it will be easier
% to exploit the computational power of the HPC cluster.

\subsection{Serial implementation}

We will now focus on the analysis of the divide and conquer algorithm
and its serial implementation.

\subsubsection{Divide}
The first step of the algorithm is the divide operation, i.e.,
split the original problem in sub problems.

The original set of points will be split in two subsets and each
subset will be split in two subsets and so on. This process will be repeated
until we have three or less points in each subset.

\subsubsection{Find the closest pair}
To find the closest pair in each subset we only have to perform
at most three comparisons and select the pair of points with the smallest
distance in the subset.

\subsubsection{Merge}
The merge operation is a crucial part in the algorithm.
The first thing to do is to take two adjacent subsets and compare the distances
found in each subset and take the pair of points with smaller distance. Then we
need to look at the boundary between the two subsets and check the distances of
the pairs where one point belongs to one subset and one point belongs to the other
subset. To check these points we first select the strip of points with $x$ distance smaller
than the current smaller distance, then we sort the selected points by the $y$ coordinate.
Finally, for each point in the strip we calculate the distance between that point and
the $7$ points around it and we compare this distance with the smaller distance found so far.
It can be proven matematically that 7 is the maximum number of points to check since, if there
would be closer points, a smaller distance would have been found in the previous step.
(Decidere se spiegare meglio e/o aggiungere l'immagine)

\subsection{Implementation}
To implement the serial algorithm we started by implementing the \verb+closest_points_divide+
function that takes an array of points already sorted by the x coordinate and split the array in two
halves and recursively computes the distances of each half using the \verb+closest_points_rec+
function. After the distances computation we use the smallest distances to select the points
near the boundary. Then we use the \verb+band_update_result+ to sort these points by the $y$ coordinate
and check the distance between each point and the next 6 points.

% state of the art
 % ~page 1 
\section{Main steps towards parallelization}
\label{sec:parallelization}

\begin{itemize}
  \item Design of the parallel solution
  \item Implementation
  \item Benchmark on the HPC@UniTrento cluster
\end{itemize}

To move from serial to parallel we decided to split the ordered
points between all the processes, then each process will perform
the serial algorithm on the subset and will find its local best distance and
finally the distances will be merged and the boundary points will be checked
 % ~page 2 
\section{Benchmarking}

The benchmarks were run on the University of Trento's High Performance Computing cluster
%\footnote{\url{https://unitrento.service-now.com/unitrento?id=unitrento_service_offering_dettaglio&category_id=68561769c33785109a8d151ce0013147&sys_id=14bb84f0c398d5104cbb7055df013133}}
.

The following combinations of parameters were considered:

\begin{table}[ht]
    \centering
    \caption{Simulation parameters}
    \begin{tabular}{l|l}
        \hline
        Parameter                     & Range                                    \\
        \hline
        CPU(s)                        & {1, 2, 4, 8, 16}                         \\
        Node(s)                       & {1, 2, 4, 8, 16}                         \\
        Packing strategy              & {pack, scatter, pack:excl, scatter:excl} \\
        Input size (number of points) & {8, 5K, 100K, 1M, 5M, 10M, 15M }         \\
        \hline
    \end{tabular}
\end{table}

The jobs were always submitted asking for 1GB of RAM and with a maximum execution time set to two minutes by default, but increased to five if we were running more than 64 processes.


\subsection{Execution time}

We saw the execution time varying like abcd
\begin{table}[ht]
    \centering
    \caption{Aggregate simulation results}
    \begin{tabular}{l|l|l|l}
        % or this to span the entire width \begin{tabularx}{\columnwidth}{X|X|X|X}
        \hline
        Nodes & Cores & Input size & Runtime (seconds) \\
        \hline
        2     & 2     & 1M         & 10                \\
        \hline
    \end{tabular}
\end{table}

\subsection{Speedup}

\subsection{Efficiency}

\subsection{Scalability}
Our implementation exhibited {strong-weak} scalability. TODO: figure this out

 % ~page 3
\section{Final Discussion}
\label{sec:final_discussion}

\subsection{Conclusions}
In this research, we produced an implementation of the closest pair of points algorithm and parallelized it on a high performance comupting cluster.
We setup Continuous Integration and Continuous Deployment pipelines that help us not only speeding up the testing, but also catching bugs in the code.

The obtained algorithm performs well, I think, yet we will insert a more formally convincing phrase here referring to the scalability it exhibited and saying that overall we think its 'good'.

\subsection{Limitations}
We assume that the point coordinates fit inside an int32.
We strived to put everything, yet we'd like to have had more time to investigate the weird behavior of the speedup shooting up at 64 processes.

\subsection{Future Works}
\label{subsec:future_works}
For sure, we'd like how an addition of the OpenMP parallelization strategy influences the outcomes of the tests.
Also, we'd like \LaTeX~to stop putting the figures where he wants, especially in the appendix.
More rigorous benchmarking, with distinctions from the various times of our program.
A comparison with \cite{wang2020parallel} would be great, but we were not able to figure it out in the time allotted.

Optimizations:
\begin{enumerate}
    \item Instead of 3 \verb|MPI_send|, we can have two by merging the first two (the best pair and the two band lenghts)
\end{enumerate} % ~page 4

\bibliographystyle{ieeetr}
\bibliography{bibliography}

\onecolumn % Switch to one column layout
\setlength\itemsep{1em} % Restore item spacing in lists

\appendix
\subsection{Full Benchmark Results}

This appendix contains the graphs with the relation between the number of processes and, respectively, the speedup, efficiency and total run time.

\begin{figure}[!ht]
    \centering
    \includesvg{../results/imgs/Speedup_pack.svg}
    \includesvg{../results/imgs/Speedup_pack:excl.svg}
    \includesvg{../results/imgs/Speedup_scatter.svg}
    \includesvg{../results/imgs/Speedup_scatter:excl.svg}
    \caption{Speedup and number of processes correlation}
    \label{fig:speedup}
\end{figure}

\begin{figure}[!ht]
    \centering
    \includesvg{../results/imgs/Efficiency_pack.svg}
    \includesvg{../results/imgs/Efficiency_pack:excl.svg}
    \includesvg{../results/imgs/Efficiency_scatter.svg}
    \includesvg{../results/imgs/Efficiency_scatter:excl.svg}
    \caption{Efficiency and number of processes correlation}
    \label{fig:efficiency}
\end{figure}

\begin{figure}[!ht]
    \centering
    \includesvg{../results/imgs/Total_time_pack.svg}
    \includesvg{../results/imgs/Total_time_pack:excl.svg}
    \includesvg{../results/imgs/Total_time_scatter.svg}
    \includesvg{../results/imgs/Total_time_scatter:excl.svg}
    \caption{Total run time and number of processes correlation}
\end{figure}

\pagebreak
\subsection{Links to resources}
\begin{itemize}
      \item Source code repository: \texttt{\url{https://github.com/civts/parallel-closest-pair}}
            \label{link:source}
      \item Input datasets: \texttt{\url{https://drive.google.com/drive/folders/12EFktu1ARw6exKn5k1AvLlHtD1Rcnx9f}}
      \item Spreadsheet with the raw data from the benchmarks -multiple bencharks, see benchmark name column-: \texttt{\url{https://docs.google.com/spreadsheets/d/1pG-KbEGtl7QHyF0bolPf2sWN__m63soeGuircwflwV0}}
            \label{link:spreadsheet}
\end{itemize}

\end{document}
