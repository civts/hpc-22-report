\section{Main steps towards parallelization}
\label{sec:parallelization}

\begin{itemize}
  \item Design of the parallel solution
  \item Implementation
  \item Benchmark on the HPC@UniTrento cluster
\end{itemize}

To move from serial to parallel we decided to split the ordered
points between all the processes, then each process will perform
the serial algorithm on its subset of points and it will find its local best distance and
finally the distances will be merged and the boundary points will be checked.

\subsection*{Implementation}
To implement the parallel algorithm we used MPI.
In particular, after ordering the points, we split them between the processes using the \verb+MPI_Scatterv+ function.
To do this we had to implement two MPI datatypes, \verb+mpi_point_type+ to represent a point and \verb+mpi_pair_of_points_type+
to represent a pair of points.

After the scatter, every process computes the closest pair using the serial algorithm described in section 2.
