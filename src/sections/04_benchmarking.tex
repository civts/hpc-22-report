\section{Benchmarking}

The benchmarks were run on the University of Trento's High Performance Computing cluster
%\footnote{\url{https://unitrento.service-now.com/unitrento?id=unitrento_service_offering_dettaglio&category_id=68561769c33785109a8d151ce0013147&sys_id=14bb84f0c398d5104cbb7055df013133}}
.

The following combinations of parameters were considered:

\begin{table}[ht]
    \centering
    \caption{Simulation parameters}
    \begin{tabular}{l|l}
        \hline
        Parameter                     & Range                                    \\
        \hline
        CPU(s)                        & {1, 2, 4, 8, 16}                         \\
        Node(s)                       & {1, 2, 4, 8, 16}                         \\
        Packing strategy              & {pack, scatter, pack:excl, scatter:excl} \\
        Input size (number of points) & {8, 5K, 100K, 1M, 5M, 10M, 15M }         \\
        \hline
    \end{tabular}
\end{table}

The jobs were always submitted asking for 1GB of RAM and with a maximum execution time set to two minutes by default, but increased to five if we were running more than 64 processes.


\subsection{Execution time}

We saw the execution time varying like abcd
\begin{table}[ht]
    \centering
    \caption{Aggregate simulation results}
    \begin{tabular}{l|l|l|l}
        % or this to span the entire width \begin{tabularx}{\columnwidth}{X|X|X|X}
        \hline
        Nodes & Cores & Input size & Runtime (seconds) \\
        \hline
        2     & 2     & 1M         & 10                \\
        \hline
    \end{tabular}
\end{table}

\subsection{Speedup}

\subsection{Efficiency}

\subsection{Scalability}
Our implementation exhibited {strong-weak} scalability. TODO: figure this out

