\section{Problem Analysis}
\label{sec:problem_analysis}

% description of the problem
The na\"ive solution to the closest pair of points problem would be to take each point
in the plane and calculate the distance between that point and all
the other points. The time complexity for this solution is $\Theta(N^2)$,
where $N\coloneqq|S|$ is the cardinality of the set $S$.

% Better solutions have been proposed, in particular in 1975
% Rabin proposed a stocastic algorithm with an expected run time
% of $O(N)$.

In 1975, Shamos and Hoey proved the lower bound for this problem to be $\Omega(N * log(N))$~\cite[§2, theorem 1]{closest_pair_definition}. An algorithm with said complexity appears
the following year from Bentley and Shamos \cite{divide_and_conq_3NlgN}.
% in a book by Preparata and Shamos \cite[§5.4]{preparata1993computational}.
A new version of that algortihm was proposed in 2006, reducing the time complexity by a factor of two \cite{ge2006improved}.

For the purposes of this research, we decided to implement the version of Bentley and Shamos, which falls into the best category of complexity whilst still keeping the base algorithm relatively simple, aiding the comprehension of the code.
% For this research, we chose to use a divide and conquer approach
% with $O(N*(logN)^2)$ time complexity. We chose it because
% divide and conquer is highly parallelizable so it will be easier
% to exploit the computational power of the HPC cluster.

\subsection{Serial Algorithm Explanation}

In order to solve the closest pair of points problem, we apply the divide et impera approach described in \cite{divide_and_conq_3NlgN}.
Our function accepts as input an array of $N$ points.

As a first step, this array is sorted in ascending order according to the X coordinate -time complexity: $\Theta(N*log(N))$-.

If the length of the array is two, then the distance between these two points is the minimum one.

If the length of the array is three, we compute the three distances between the pairs of points and return the minimum one.

If the length is greater than three, we:
\begin{inlinelist}
    \item divide the points by splitting them in two halves of length $\lfloor \frac{N}{2} \rfloor$ and $\lceil \frac{N}{2} \rceil$,
    \item find the closest pair in each half by recursively calling the closest pair of points function on them,
    \item merge the two results to find the closest pair in the starting array. \label{step:merge}
\end{inlinelist}

Let us focus our attention on step \ref{step:merge}: the merge.

In order to combine the two partial results, we first determine which among the the two pairs has the smallest distance, let that be $\delta$.

Next, we condiser a vertical band of points of thickness $2\delta$ centered on the line we used to split the points -i.e., on the median of the X coordinates of the rightmost point of the left half and the leftmost point of the right half-.

After sorting the points in that band according to their Y coordinate, we search if they contain a pair closer than $\delta$.

To do so, we compute the distance between each point of the band and the seven points that come after it,
%fifteen/eight/sixteen https://youtu.be/xi-WF07rAQw?t=1673 https://archive.org/details/AlgorithmDesign1stEditionByJonKleinbergAndEvaTardos2005PDF/page/n253/mode/2up
updating our closest pair if we find a couple of points that are closer than the minimum distance \cite{lingqi}.

%(Decidere se spiegare meglio e/o aggiungere l'immagine)

\subsection{Existing Parallel Implementations}
We searched online for existing implementations that use OpenMPI to parallelize the algorithm, yet found only one \cite{gh1} that generated the points randomly every time and appeared to have bad memory management and another \cite{gh2_fortran} which continued to go in segmentation fault.
