\section{Introduction}
\label{sec:introduction}

% what we are talking about
The \textit{closest pair of points} problem consists
in finding the two nearest points in a set, $S$, according to a given distance function, $d$\cite{closest_pair_definition}.
These points can live in a space with an arbirary number of dimensions
and the choice of the distance function is also arbitrary.
For the purposes of this research, we focus on the two-dimensional problem, using the Euclidean distance and integer coordinates.
With minimal adjustments, the code can be adapted to solve a higher dimensionality problem, to adopt a different distance function or to use, e.g., floating point coordinates.

% objective
This research aims to study the implications of running an algorithm to solve the closest pair of points problem in parallel on a cluster for High Performance Computing.

% structure of the paper
This paper is structured as follows:
\begin{inlinelist}
    \item \Cref{sec:problem_analysis} describes the state of the art for the closest pair of points problem.
    \item \Cref{sec:parallelization} treats how we designed and produced our parallel implementation.
    \item \Cref{sec:final_discussion} provides an evaluation of the obtained results using benchmarking techniques and our conclusions on the project as a whole.
\end{inlinelist}
