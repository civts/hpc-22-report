\section{Introduction}
\label{sec:introduction}

The \textit{parallel closest pair of points} problem consists
in finding the smaller distance between two points in the plane.
We will focus on the 2-dimensional problem, however, solutions for
the $d$ dimensional space exist.

The naive solution to this problem would be to take each point
in the plane and calculate the distance between that point and all
the other points. The time complexity for this solution is $O(N^2)$
where $N$ is the number of points in the space.
Better solutions have been proposed, in particular in 1976
Rabin proposed a randomized algorithm with an expected run time
of $O(N)$ and in 1979 Fortune and Hopcroft proposed a
deterministic $O(N log log N)$ solution \cite{Fortune_Hopcroft}
assuming that the floor operation takes constant time.

In this project we will explore a divide and conquer approach
with $O(N(logN)^2)$ time complexity. We chose it because
divide and conquer is highly parallelizable so it will be easier
to exploit the computational power of the HPC cluster.
