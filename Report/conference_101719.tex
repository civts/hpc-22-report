\documentclass[conference]{IEEEtran}
\IEEEoverridecommandlockouts
% The preceding line is only needed to identify funding in the first footnote. If that is unneeded, please comment it out.
\usepackage{cite}
\usepackage{amsmath,amssymb,amsfonts}
\usepackage{algorithmic}
\usepackage{graphicx}
\usepackage{textcomp}
\usepackage{xcolor}
\def\BibTeX{{\rm B\kern-.05em{\sc i\kern-.025em b}\kern-.08em
    T\kern-.1667em\lower.7ex\hbox{E}\kern-.125emX}}
\begin{document}

\title{High Performance Computing for Data Science Project\\
{\footnotesize Prof. Fiore Sandro Luigi}}

\author{\IEEEauthorblockN{Guidolin Davide}
    \IEEEauthorblockA{davide.guidolin@studenti.unitn.it}
    \and
    \IEEEauthorblockN{Zanolli Giacomo}
    \IEEEauthorblockA{giacomo.zanolli@studenti.unitn.it}
}

\maketitle

\begin{abstract}
    This is the report for the High Performance Computing for Data Science course project.
    
    The goal of the project was to implement a parallel solution for the \textit{closest pair of points} problem
    and evaluate it on the HPC@UniTrento cluster.
    
    In the following sections we will present the problem and we will describe some serial solutions
    for it. Then we will present how we implemented a parallel solution and finally we will present and
    discuss the evaluation of the parallel solution on the HPC cluster.
\end{abstract}

\section{Introduction}
\input{introduction}
\section{Problem analysis}
Serial implementation

We will now focus on the analysis of the divide and conquer algorithm 
and its serial implementation.

\subsection{Divide}
The first step of the algorithm is the divide operation, i.e.
split the original problem in sub problems. 

The original set of points will be split in two subsets and each
subset will be split in two subsets and so on. This process will be repeated
until we have three or less points in each subset.
\subsection{Find the closest pair}
To find the closest pair in each subset we only have to perform
at most three comparisons and select the pair of points with the smallest
distance in the subset.
\subsection{Merge}
The merge operation is a crucial part in the algorithm.
The first thing to do is to take two adjacent subsets and compare the distances
found in each subset and take the pair of points with smaller distance. Then we 
need to look at the boundary between the two subsets and check the distances of 
the pairs where one point belongs to one subset and one point belongs to the other
subset. To check these points we first select the strip of points with $x$ distance smaller
than the current smaller distance, then we sort the selected points by the $y$ coordinate.
Finally, for each point in the strip we calculate the distance between that point and 
the $7$ points around it and we compare this distance with the smaller distance found so far.


\subsection{Implementation}

\section{Main steps towards parallelization}
\input{parallelization}
\section{Final discussion}
\input{final_discussion}

\bibliographystyle{ieeetr}
\bibliography{mybib}

\end{document}
