\documentclass[conference]{IEEEtran}
\IEEEoverridecommandlockouts
% The preceding line is only needed to identify funding in the first footnote. If that is unneeded, please comment it out.
\usepackage{cite}
\usepackage{amsmath,amssymb,amsfonts}
\usepackage{algorithmic}
\usepackage{graphicx}
\usepackage{textcomp}
\usepackage{xcolor}
\def\BibTeX{{\rm B\kern-.05em{\sc i\kern-.025em b}\kern-.08em
    T\kern-.1667em\lower.7ex\hbox{E}\kern-.125emX}}
\begin{document}

\title{High Performance Computing for Data Science Project\\
{\footnotesize Prof. Fiore Sandro Luigi}}

\author{\IEEEauthorblockN{Guidolin Davide}
    \IEEEauthorblockA{davide.guidolin@studenti.unitn.it}
    \and
    \IEEEauthorblockN{Zanolli Giacomo}
    \IEEEauthorblockA{giacomo.zanolli@studenti.unitn.it}
}

\maketitle

\begin{abstract}
    This is the report for the High Performance Computing for Data Science course project.
    
    The goal of the project was to implement a parallel solution for the \textit{closest pair of points} problem
    and evaluate it on the HPC@UniTrento cluster.
    
    In the following sections we will present the problem and we will describe some serial solutions
    for it. Then we will present how we implemented a parallel solution and finally we will present and
    discuss the evaluation of the parallel solution on the HPC cluster.
\end{abstract}

\section{Introduction}
The \textit{parallel closest pair of points} problem consists 
in finding the smaller distance between two points in the plane.
We will focus on the 2-dimensional problem, however, solutions for 
the $d$ dimensional space exist.

The naive solution to this problem would be to take each point 
in the plane and calculate the distance between that point and all 
the other points. The time complexity for this solution is $O(N^2)$
where $N$ is the number of points in the space.
Better solutions have been proposed, in particular in 1976 
Rabin proposed a randomized algorithm with an expected run time 
of $O(N)$ and in 1979 Fortune and Hopcroft proposed a 
deterministic $O(N log log N)$ solution \cite{Fortune_Hopcroft}
assuming that the floor operation takes constant time.

In this project we will explore a divide and conquer approach 
with $O(N(logN)^2)$ time complexity. We chose it because
divide and conquer is highly parallelizable so it will be easier
to exploit the computational power of the HPC cluster.
\section{Problem analysis}
Serial implementation

We will now focus on the analysis of the divide and conquer algorithm 
and its serial implementation.

\subsection{Divide}
The first step of the algorithm is the divide operation, i.e.
split the original problem in sub problems. 

The original set of points will be split in two subsets and each
subset will be split in two subsets and so on. This process will be repeated
until we have three or less points in each subset.

\subsection{Find the closest pair}
To find the closest pair in each subset we only have to perform
at most three comparisons and select the pair of points with the smallest
distance in the subset.

\subsection{Merge}
The merge operation is a crucial part in the algorithm.
The first thing to do is to take two adjacent subsets and compare the distances
found in each subset and take the pair of points with smaller distance. Then we 
need to look at the boundary between the two subsets and check the distances of 
the pairs where one point belongs to one subset and one point belongs to the other
subset. To check these points we first select the strip of points with $x$ distance smaller
than the current smaller distance, then we sort the selected points by the $y$ coordinate.
Finally, for each point in the strip we calculate the distance between that point and 
the $7$ points around it and we compare this distance with the smaller distance found so far.
It can be proven matematically that 7 is the maximum number of points to check since, if there 
would be closer points, a smaller distance would have been found in the previous step.
(Decidere se spiegare meglio e/o aggiungere l'immagine)

\subsection{Implementation}
To implement the serial algorithm we started by implementing the \verb+closest_points_divide+ 
function that takes an array of points already sorted by the x coordinate and split the array in two 
halves and recursively computes the distances of each half using the \verb+closest_points_rec+ 
function. After the distances computation we use the smallest distances to select the points
near the boundary. Then we use the \verb+band_update_result+ to sort these points by the $y$ coordinate
and check the distance between each point and the next 6 points.

\section{Main steps towards parallelization}
\begin{itemize}
  \item Design of the parallel solution
  \item Implementation
  \item Benchmark on the HPC@UniTrento cluster
\end{itemize}

\section{Final discussion}
Conclusions

\bibliographystyle{ieeetr}
\bibliography{mybib}

\end{document}
